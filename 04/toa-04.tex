\NeedsTeXFormat{LaTeX2e}
%\PassOptionsToClass{handout}{beamer}
\documentclass{beamer}
\usepackage{beamerPack}
\usepackage[boxed,ruled,vlined]{algorithm2e}
\usepackage[04]{../lecture}
\subtitle{}
\begin{document}

\begin{frame}[fragile]{}
\titlepage
\end{frame}

\section{algorithm design}		%%%%%%%%
\subsection{}

\begin{frame}[fragile]{素朴なアルゴリズムと巧妙なアルゴリズムの違い}{}
a
\end{frame}

\begin{frame}[fragile]{素数の計算}{}
a
\end{frame}

\begin{frame}[fragile]{アルゴリズムの改良}{素数}
a
\end{frame}

\begin{frame}[fragile]{アルゴリズムの改良}{フィボナッチ数}
a
\end{frame}

\begin{frame}[fragile]{問題の再帰構造の次元を落とす(線形化)}{}
a
\end{frame}

\section{dynamic programming}		%%%%%%%%
\subsection{}

\begin{frame}[fragile]{動的計画法}{}

\begin{block}{動的計画法(dynamic programming)}
\begin{itemize}%\itemsep8pt
\item 再帰的に分割された部分問題の(最適)解から元問題の(最適)解を構成
\item 部分問題の重複性の利用
\end{itemize}
\end{block}

\footnotetext{線形計画法は不等式に関する問題}

\end{frame}

\begin{frame}[fragile]{例題1:Pascalの三角形}{}
a
\end{frame}

\begin{frame}[fragile]{例題2:飛行経路}{}
a
\end{frame}


\section{for programmers}		%%%%%%%%
\subsection{}

\begin{frame}[fragile]{数学的解析の力}{}

\[
Fib(n) = \frac{1}{\sqrt{5}}
\left\{
\left(\frac{1 + \sqrt{5}}{2}\right)^n
-
\left(\frac{1 - \sqrt{5}}{2}\right)^n
\right\}
\]

\end{frame}

\begin{frame}[fragile]{ふかしぎおねえさん}{}
a
\end{frame}

\begin{frame}[fragile]{メモ化}{}
a
\end{frame}

\begin{frame}[fragile]{自動メモ化}{}
a
\end{frame}

\end{document}
